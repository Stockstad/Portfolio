\documentclass{TDP003mall}



\newcommand{\version}{Version 1.1}
\author{Johan Törner, \url{johto839@student.liu.se}\\
  Linus Nordin, \url{linno988@student.liu.se}}
\title{Projektplan}
\date{2023-09-18}
\rhead{Johan Törner\\
Linus Nordin\\}



\begin{document}
\projectpage
\section{Revisionshistorik}
\begin{table}[!h]
\begin{tabularx}{\linewidth}{|l|X|l|}
\hline
Ver. & Revisionsbeskrivning & Datum \\\hline
1.0 & Skapad & 23-09-18\\\hline
\end{tabularx}
\end{table}


\section{Inledning}
Projektets syfte är att skapa en portfolio som innehåller de projekt som skapas under IP-programmet. 
Portfolion är en webbsida vairfrån en användare kan söka på projekt som ligger i en databas. Ansvariga för projektet 
är Johan Törner och Linus Nordin. Projektet kommer att följa de riktlinjer som satts och kommer att innehålla de delar
som specificerats. 


\section{Arbetssätt}

\subsection{Tekniker}
De tekniker som kommer att användas i projektets gång är följande: Python, HTML, CSS, TailwindCSS, Flask, Jinja2, JSON, Git samt Latex. 

\subsubsection{Backend}
På backend används Python, Flask, JSON samt Jinja2. Flask är ett lightweight-framework för webben baserat på Python.
Tillsammans med Jinja2 kan vi rendera HTML direkt från servern. Dessutom använder vi JSON för att spara informaton om varje projekt vairfrån
användaren sedan kan hämta dem.

\subsubsection{Frontend}
För frontend används HTML och CSS, med vilket Tailwind används som CSS+framework.
Där som tidigare nämnt HTML renderas från servern av Jinja2 med hjälp av Flask. CSS används för att styla HTML. 


\subsubsection{Övrigt}
Git används för versionhantering. Latex används för dokumentskrivning. 

\subsection{Kommunikation}
Kommunikation sker huvudsakligen via Discord, om en inte är närvarande på campus förstås. När man träffas på morgonen upptas dagens 
planering och bestämmer man som gör vad. Ytterligare information om detta finns under \textbf{\textit{uppdelning av arbetsuppgifter}}.

\subsection{Upplägg}
Huvudsakligen utförs arbetet på plats på campus, med undantag för tillfällen då jobb måste utföras under helgen (till följd av felaktigheter i planering)
alternativt när någon är sjuk. För mer information om arbetssättet, se \textbf{\textit{gruppkontraktet}}. För mer specifik information angående tidsplaneringen, 
se \textbf{\textit{planeringen}}. 

\subsection{Uppdelning av arbetsuppgifter}
Huvudsakligen delas arbetet upp. Eftersom arbetet sker på plats går det enkelt att gå igenom andras kod, alternativ svara på frågor. 
Vid behov, men huvudsakligen vid starten på dagen, planeras vad som ska göras samt vem som ska göra vad för dagen. Huvudsakligen utgår man från
planerningen, men med att olika moment kan ha uppskattats fel planeras dessa om. 

\section{Planering}
Planeringens huvudstyfte är att projektet ska hållas i rätt tidsfas och att projektmedlemmarna alltså har en överblick över när och vad som ska vara klart.
\subsection{Givna deadlines}
Måndag 04-09: Gruppkontraktet\\
Tisdag 12-09: Tidsplanen\\
Fredag 15-09: LoFi-prototypen\\
Torsdag 21-09: Installationsmanualen v1, Projektplan utkast\\
Torsdag 28-09: Installationsmanualen, Projektplan (brister återgärdade)\\
Fredag 29-09: Datalagret\\
Torsdag 12-10: Publicering av Portfolio, Systemdokumentation v1\\
Torsdag 19-10: Reflektionsdokument, Systemdokumentation (brister återgärdade)\\

\subsection{Milstolpar}
Måndag: 25-09: Datalagrets API-funktioner klara.\\
Onsdag 27-09: Mainprogram till API klar.\\
Onsdag 04-10: Möjlighet att söka på hemsidan och få upp projekt.\\
Fredag 06-10: Adminpanel klar. (Låg prio)\\




\end{document}
